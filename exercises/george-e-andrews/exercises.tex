\documentclass[11pt]{article}
\usepackage{amssymb}
\usepackage{amsthm}
\usepackage{amsmath}
\usepackage{enumerate}
\usepackage{fancyhdr}
\usepackage[bottom=0.9in,top=0.9in]{geometry}
\usepackage{hyperref}
\hypersetup{colorlinks=true,linktoc=all,linkcolor=blue}
\usepackage{mathtools}

\begin{document}
\pagestyle{fancy}
\fancyhead{}
\fancyhead[C]{\textbf{Exercises from George E. Andrews' Number Theory}}
\tableofcontents
\newpage
\fancyhead[L]{\leftmark}
\fancyhead[C]{}
\fancyhead[R]{\rightmark}

\section{Basis Representation}
% 1.1
\subsection{Principle of Mathematical Induction}
% 1.1.1
\subsubsection{}
Prove that
\[
	\sum_{i=0}^{n} i^2 = 1^2 + 2^2 + 3^2 + ... + n^2 = \frac{n(n+1)(2n+1)}{6}.
\]
\begin{proof}
	If $n=1$, then
	\[
		\sum_{i=0}^{1} i^2 = 1 = \frac{1(1+1)(2+1)}{6}.
	\]
	Thus, the identity holds for $n=1$. Now assume that
	\[
		\sum_{i=0}^{n} i^2 = \frac{n(n+1)(2n+1)}{6}
	\]
	for some $n \geq 1$, and compute
	\begin{align*}
		\sum_{i=0}^{n+1} i^2 & = \frac{n(n+1)(2n+1)}{6} + (n+1)^2 \\
				     & = (n+1) \left[ \frac{n(2n+1)}{6} + (n+1) \right] \\
				     & = (n+1) \left[ \frac{n(2n+1) + 6(n+1)}{6} \right] \\
				     & = (n+1) \left[ \frac{2n^2 + n + 6n + 6}{6} \right] \\
				     & = (n+1) \left[ \frac{2n^2 + 4n + 3n + 6}{6} \right] \\
				     & = (n+1) \left[ \frac{2n(n+2) + 3(n+2)}{6} \right] \\
				     & = (n+1) \left[ \frac{(n+2)(2n+3)}{6} \right] \\
				     & = \frac{(n+1)(n+2)(2n+3)}{6} \\
				     & = \frac{(n+1)((n+1)+1)(2(n+1)+1)}{6},
	\end{align*}
	so the identity also holds for $n+1$. Thus, by the principle of mathematical induction,
	the identity holds for all $n \geq 1$, where $n \in \mathbb{N}$.
\end{proof}

\newpage

% 1.1.2
\subsubsection{}
Prove that
\[
	\sum_{i=0}^{n} i^3 = 1^3 + 2^3 + 3^3 + ... + n^3 = (1+2+3+...+n)^3.
\]
\begin{proof}
	If $n=1$, then
	\[
		\sum_{i=0}^{1} i^3 = 1 = (1)^3.
	\]
	Thus, the identity holds for $n=1$. Now assume that
	\[
		\sum_{i=0}^{n} i^3 = (1+2+3+...+n)^3
	\]
	for some $n \geq 1$, and compute 
	\begin{align*}
		\sum_{i=0}^{n+1} i^3 & = \sum_{i=0}^{n} i^3 + (n+1)^3\\
				     & = (1+2+3+...+n)^3 + (n+1)^3 \\
				     & = \left( \frac{n(n+1)}{2} \right)^3 + (n+1)^3 \\
				     & = \left( \left( \frac{n(n+1)}{2} \right) + (n+1) \right)^3 \\
				     & = \left( \frac{n(n+1)+2(n+1)}{2} \right)^3 \\
				     & = \left( \frac{(n+1)(n+2)}{2} \right)^3 \\
				     & = \left( \frac{(n+1)((n+1)+1)}{2} \right)^3 \\
				     & = (1+2+3+...+(n+1))^3,
	\end{align*}
	so the identity also holds for $n+1$. Thus, by the principle of mathematical induction,
	the identity holds for all $n \geq 1$, where $n \in \mathbb{N}$.
\end{proof}

\newpage

\setcounter{subsubsection}{3}

% 1.1.4
\subsubsection{}
Prove that
\[
	\sum_{i=0}^{n} i(i+1) = 1\cdot2 + 2\cdot3 + 3\cdot4 + ... + n(n+1) = \frac{n(n+1)(n+2)}{3}.
\]
\begin{proof}
	If $n=1$, then
	\[
		\sum_{i=0}^{1} i(i+1) = 2 = \frac{1(2)(3)}{3}.
	\]
	Thus, the identity holds for $n=1$. Now assuming that
	\[
		\sum_{i=0}^{n} i(i+1) = \frac{n(n+1)(n+2)}{3}
	\]
	for some $n \geq 1$, and compute
	\begin{align*}
		\sum_{i=0}^{n+1} i(i+1) & = \left( \sum_{i=0}^{n} i(i+1) \right) + (n+1)((n+1)+1) \\
				        & = \frac{n(n+1)(n+2)}{3} + (n+1)(n+2) \\
				        & = \frac{n(n+1)(n+2) + 3(n+1)(n+2)}{3} \\
				        & = \frac{(n+1)(n+2)(n+3)}{3} \\
				        & = \frac{(n+1)((n+1)+1)((n+1)+2)}{3},
	\end{align*}
	so the identity also holds for $n+1$. Thus, by the principle of mathematical induction,
	the identity holds for all $n \geq 1$, where $n \in \mathbb{N}$.
\end{proof}

\newpage

% 1.1.5
\subsubsection{}
Prove that
\[
	\sum_{i=1}^{n} (2i-1) = 1 + 3 + 5 + ... + (2n-1) = n^2.
\]
\begin{proof}
	If $n=1$, then
	\[
		\sum_{i=1}^{1} (2i-1) = 1 = 1^2.
	\]
	Thus, the identity holds for $n=1$. Now assume that
	\[
		\sum_{i=1}^{n} (2i-1) = n^2
	\]
	for some $n \geq 1$, and compute
	\begin{align*}
		\sum_{i=1}^{n+1} (2i-1) & = \left( \sum_{i=1}^{n} (2i-1) \right) + \left(2(n+1)-1\right) \\
				        & = n^2 + 2(n+1) - 1 \\
				        & = n^2 + 2n + 1 \\
				        & = (n+1)(n+1) \\
				        & = (n+1)^2,
	\end{align*}
	so the identity also holds for $n+1$. Thus, by the principle of mathematical induction,
	the identity holds for all $n \geq 1$, where $n \in \mathbb{N}$.
\end{proof}

\end{document}
