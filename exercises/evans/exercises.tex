\documentclass[11pt]{article}
\usepackage{amssymb}
\usepackage{amsthm}
\usepackage{amsmath}
\usepackage{enumerate}
\usepackage{fancyhdr}
\usepackage[bottom=0.9in,top=0.9in]{geometry}
\usepackage{hyperref}
\hypersetup{colorlinks=true,linktoc=all,linkcolor=blue}
\usepackage{mathtools}

\newcommand{\defeq}{\vcentcolon=}

\begin{document}
\pagestyle{fancy}
\fancyhead[L]{}
\fancyhead[R]{}
\fancyhead[C]{\textbf{Exercises from Evans Partial Differential Equations}}
\tableofcontents
\newpage
\fancyhead[L]{\leftmark}
\fancyhead[R]{\rightmark}
\fancyhead[C]{}
\setcounter{section}{4}

\section{Sobolev Spaces}
\subsection{}
Suppose $k \in \{0,1,\dots\}$, $0 < \gamma < 1$. Prove $C^{k,\gamma}(\overline{U})$ is a Banach space.
\begin{proof}
A Banach space is a \textit{complete normed linear space}, so we need to show that $C^{k,\gamma}(\overline{U})$
is each of the following:
	\begin{enumerate}[1.]
		\item A Linear space.
		\item A Normed linear space.
		\item A Complete space.
	\end{enumerate}
Much of the proof will rely on the fact that we know $C(\overline{U})$ is a Banach space itself.
	\begin{enumerate}[1.]
		\item Suppose $u,v,w \in C^{k,\gamma}(\overline{U})$ and $a,b \in \mathbb{R}$.
			It follows that $u,v,w \in C^k(\overline{U})$ and that the H\"{o}lder norm of
			each $u$, $v$, and $w$ is finite. With this knowledge we will loosely prove
			the linear space axioms as follows:
			\begin{enumerate}[(i)]
				\item $u+v \in C^{k,\gamma}(\overline{U})$ because $\|u+v\|_{C^{k,\gamma}(\overline{U})} < \infty$,
					which follows from the triangle inequality of norms.
				\item $u+v = v+u$ because $u,v \in C^k(\overline{U})$, which we know is itself a linear space.
				\item $u+(v+w) = (u+v)+w$ following the same argument as (ii).
				\item $C^k(\overline{U}) \subset C^{k,\gamma}(\overline{U})$ implies $\exists \, \vec{0} \in C^{k,\gamma}(\overline{U})$
					such that $u + \vec{0} = u$.
				\item Following the same argument as (iv) we get that $\forall \, u \in C^{k,\gamma}(\overline{U}) \, \exists \, (-u)$
					such that $u+(-u) = \vec{0}$.
				\item $a(bu) = (ab)u$ following the same argument as (ii).
				\item $a(u + v) = au + av$ following the same argument as (ii).
				\item $(a+b)u = au + bu$ following the same argument as (ii).
				\item Following the same argument as (iv) we get that $\exists \, \vec{1} \in C^{k,\gamma}(\overline{U})$
					such that $\vec{1} \cdot u = u$.
				\item Following the same argument as (iv) we get that $\exists \, \vec{0} \in C^{k,\gamma}(\overline{U})$
					such that $\vec{0} \cdot u = \vec{0}$.
			\end{enumerate}
			These axioms prove that $C^{k,\gamma}(\overline{U})$ is indeed a linear space.
		\item Using the fact that $C(\overline{U})$ is itself a normed linear space, with the norm
			\[\|u(x)\|_{C(\overline{U})} \defeq \sup_{x \in U}{|u(x)|},\]
			we can prove the normed linear space axioms as follows:
			\begin{enumerate}[(i)]
				\item Suppose $u,v \in C^{k,\gamma}(\overline{U})$ and that $\alpha$ is a fixed multi-index
					such that $|\alpha| \leq k$. It then follows that
					\begin{align}
						\|u+v\|_{C^{k,\gamma}(\overline{U})} &= \sum_{|\alpha| \leq k}\|D^{\alpha}(u+v)\|_{C(\overline{U})}
											+ \sum_{|\alpha| = k}[D^{\alpha}(u+v)]_{C^{0,\gamma}(\overline{U})} \\
						&\leq \sum_{|\alpha| \leq k}(\|D^{\alpha}u\|_{C(\overline{U})} + \|D^{\alpha}v\|_{C(\overline{U})})
						    + \sum_{|\alpha| = k}[D^{\alpha}(u+v)]_{C^{0,\gamma}(\overline{U})}\\
						&\leq \sum_{|\alpha| \leq k}(\|D^{\alpha}u\|_{C(\overline{U})} + \|D^{\alpha}v\|_{C(\overline{U})})
						    + \sum_{|\alpha| = k}([D^{\alpha}u]_{C^{0,\gamma}(\overline{U})} + [D^{\alpha}v]_{C^{0,\gamma}(\overline{U})})\\
						&= \sum_{|\alpha| \leq k}\|D^{\alpha}u\|_{C(\overline{U})} + \sum_{|\alpha| \leq k}\|D^{\alpha}v\|_{C(\overline{U})}
						    + \sum_{|\alpha| = k}[D^{\alpha}u]_{C^{0,\gamma}(\overline{U})} + \sum_{|\alpha| = k}[D^{\alpha}v]_{C^{0,\gamma}(\overline{U})}\\
						&= (\sum_{|\alpha| \leq k}\|D^{\alpha}u\|_{C(\overline{U})} +  \sum_{|\alpha| = k}[D^{\alpha}u]_{C^{0,\gamma}(\overline{U})})
						    + (\sum_{|\alpha| \leq k}\|D^{\alpha}v\|_{C(\overline{U})} + \sum_{|\alpha| = k}[D^{\alpha}v]_{C^{0,\gamma}(\overline{U})})\\
					        &= \|u\|_{C^{k,\gamma}(\overline{U})} + \|v\|_{C^{k,\gamma}(\overline{U})}.
					\end{align}
				\item Suppose $m \in \mathbb{R}$, $u \in C^{k,\gamma}(\overline{U})$, and that $\alpha$ is a multi-index such that
					$|\alpha| \leq k$. It then follows that
					\begin{align}
						\|mu\|_{C^{k,\gamma}(\overline{U})} &= \sum_{|\alpha| \leq k}\|mD^{\alpha}u\|_{C^k(\overline{U})}
						                                      +\sum_{|\alpha| = k}[mD^{\alpha}u]_{C^{0,\gamma}(\overline{U})} \\
										    &= |m|\sum_{|\alpha| \leq k}\|D^{\alpha}u\|_{C^k(\overline{U})}
						                                      +|m|\sum_{|\alpha| = k}[D^{\alpha}u]_{C^{0,\gamma}(\overline{U})} \\
										    &= |m|(\sum_{|\alpha| \leq k}\|D^{\alpha}u\|_{C^k(\overline{U})}
										      +\sum_{|\alpha| = k}[D^{\alpha}u]_{C^{0,\gamma}(\overline{U})}) \\
										    &= |m|\|u\|_{C^{k,\gamma}(\overline{U})}.
					\end{align}
				\item We now want to show that $\|u\|_{C^{k,\gamma}(\overline{U})} \geq 0$, $\forall \, u \in C^{k,\gamma}(\overline{U})$.
					So suppose $u \in C^{k,\gamma}(\overline{U})$, and from (ii) we know that
					\[\|-u\|_{C^{k,\gamma}(\overline{U})} = \|(-1)u\|_{C^{k,\gamma}(\overline{U})} = \|u\|_{C^{k,\gamma}(\overline{U})},\]
					which implies that
					\begin{align*}
						\|0\|_{C^{k,\gamma}(\overline{U})} &= \|u + (-u)\|_{C^{k,\gamma}(\overline{U})} \\
										   &\leq \|u\|_{C^{k,\gamma}(\overline{U})} + \|-u\|_{C^{k,\gamma}(\overline{U})} \\
										   &= \|u\|_{C^{k,\gamma}(\overline{U})} + \|u\|_{C^{k,\gamma}(\overline{U})}\\
										   &= 2\|u\|_{C^{k,\gamma}(\overline{U})}.
					\end{align*}
					Since $u$ is arbitrary we can say
					\[\|0\|_{C^{k,\gamma}(\overline{U})} \leq 2\|0\|_{C^{k,\gamma}(\overline{U})},\]
					which further implies
					\[0 \leq \|0\|_{C^{k,\gamma}(\overline{U})}\]
					when we substract through by $\|0\|_{C^{k,\gamma}(\overline{U})}$. Combining inequalities from above gives us
					\[0 \leq \|0\|_{C^{k,\gamma}(\overline{U})} \leq 2\|u\|_{C^{k,\gamma}(\overline{U})},\]
					where dividing through by 1/2 results in
					\[0 \leq \|u\|_{C^{k,\gamma}(\overline{U})}.\]
				\item We finally want to show that $\|u\|_{C^{k,\gamma}(\overline{U})} = 0$ if and only if $u = 0$.
					Firstly, suppose $u \in C^{k,\gamma}(\overline{U})$ and that $\|u\|_{C^{k,\gamma}(\overline{U})} = 0$.
					Letting $\alpha$ be some fixed multi-index such that $|\alpha| \leq k$, it then follows that
					\[\|u\|_{C^{k,\gamma}(\overline{U})} = \sum_{|\alpha| \leq k}\|D^{\alpha}u\|_{C(\overline{U})} + \sum_{|\alpha| = k}[D^{\alpha}u]_{C^{0,\gamma}(\overline{U})} = 0.\]
					By definition we know that each $\|D^{\alpha}u\|_{C(\overline{U})} \geq 0$, and also by definition we know that
					\[[D^{\alpha}u]_{C^{0,\gamma}(\overline{U})} = \sup_{x,y \in U, x \not= y}{\left\{\frac{|D^{\alpha}u(x) - D^{\alpha}u(y)|}{|x-y|^{\gamma}}\right\}} \geq 0.\]
					Therefore, if $\|D^{\alpha}u\|_{C^{k,\gamma}(\overline{U})} = 0$, then it must be that each
					$\|D^{\alpha}u\|_{C(\overline{U})} = 0$ and each $[D^{\alpha}u]_{C^{0,\gamma}(\overline{U})} = 0$.
					This further implies from the definitions that $D^{\alpha}u = 0$ implying $u = 0$, $\forall \, x \in U$.
					Conversely, suppose $u = 0$. Then by (ii) we can see that
					\begin{align*}
						\|u\|_{C^{k,\gamma}(\overline{U})} &= \|0\|_{C^{k,\gamma}(\overline{U})} \\
										   &= \|0\cdot 0\|_{C^{k,\gamma}(\overline{U})} \\
										   &= |0|\|0\|_{C^{k,\gamma}(\overline{U})} \\
										   &= 0.
					\end{align*}
			\end{enumerate}
			These axioms prove that $C^{k,\gamma}(\overline{U})$ is indeed a normed linear space.
		\item 
	\end{enumerate}
\end{proof}

\newpage

\setcounter{subsection}{2}
\subsection{}
Denote by $U$ the open square $\{x \in \mathbb{R}^2 \, | \, |x_1| < 1, |x_2| < 1\}$. Define

\[
u(x) =
\begin{cases} 
	1 - x_1, & \text{if } x_1 > 0, |x_2| < x_1 \\
	1 + x_1, & \text{if } x_1 < 0, |x_2| < -x_1 \\
	1 - x_2, & \text{if } x_2 > 0, |x_1| < x_2 \\
	1 + x_2, & \text{if } x_2 < 0, |x_1| < -x_2.
\end{cases}
\]
For which $1 \leq p \leq \infty$ does $u$ belong to $W^{1,p}(U)$.
\begin{proof}
In progress...
\end{proof}

\setcounter{subsection}{4}
\subsection{}
Let $U$, $V$ be open sets, with $V\subset\subset U$. Show there exists a smooth function $\zeta$ such that
$\zeta \equiv 1$ on $V$, $\zeta = 0$ near $\partial U$. (Hint: Take $V \subset\subset W \subset\subset U$ and mollify $\chi_W$.)
\begin{proof}
In progress...
\end{proof}

\subsection{}
Assume $U$ is bounded and $U \subset\subset \bigcup_{i=1}^{N}V_i$. Show there exist $C^{\infty}$ functions $\zeta_i$ ($i=1,\dots,N$)
such that
\[
\begin{cases}
	0 \leq \zeta_i \leq 0, \text{ supp } \zeta_i \subset V_i \,(i=1,\dots,N)\\
	\sum_{i=1}^{N} \zeta_i \text{ on } U.
\end{cases}
\]
\begin{proof}
In progress...
\end{proof}

\end{document}
