\documentclass[11pt]{article}
\usepackage{amssymb}
\usepackage{amsthm}
\usepackage{amsmath}
\usepackage{enumerate}
\usepackage{fancyhdr}
\usepackage[bottom=0.9in,top=0.9in]{geometry}
\usepackage{hyperref}
\hypersetup{colorlinks=true,linktoc=all,linkcolor=blue}
\usepackage{mathtools}

\begin{document}
\pagestyle{fancy}
\fancyhead{}
\fancyhead[C]{\textbf{Exercises from Zachmanoglou \& Dale's Intro to PDEs with Applications}}
\tableofcontents
\fancyhead[L]{\leftmark}
\fancyhead[C]{}
\fancyhead[R]{\rightmark}
\newpage
\section{Some Concepts from Calculus and ODEs}
\subsection{}
Prove that a closed set contains all of its boundary points while an open set contains none of its boundary points.
\begin{proof}
Let $A \subset \mathbb{R}^n$ be a closed set. This implies that $A$ contains all
of its limit points. Now suppose the point $x \in \mathbb{R}^n$ is a boundary
point of the set $A$. This implies that every ball with center at $x$ contains
points
of $A$ and points of $A^c$. It then follows, from the definition of a limit point,
that $x$ is itself a limit point of $A$. Then since we know $A$ contains all
of its limit points then it follows that $x \in A$, as desired. \newline
\indent Now let $A \subset \mathbb{R}^n$ be an open set. This implies that every
point of $A$ is an interior point, futher implying that $A$ contains none of its
limit points. Since we just showed that boundary points are also limit points,
it must be that $A$ contains none of its boundary points, as desired.
\end{proof}
\end{document}
